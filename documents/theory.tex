\documentclass[a4paper, 11pt]{article}
\usepackage{natbib}
\usepackage{amsmath}
\usepackage{bm}
\usepackage{amssymb}

\setlength\topmargin{-0.25in} \setlength\oddsidemargin{-0.25in}
\setlength\textheight{9.0in} \setlength\textwidth{7.0in}
\setlength\columnsep{0.375in} \newlength\titlebox \setlength\titlebox{2.25in}
\setlength\headheight{0pt}  \setlength\headsep{0pt}

\parindent 10pt
\topsep 4pt plus 1pt minus 2pt
\partopsep 1pt plus 0.5pt minus 0.5pt
\itemsep 0.5pt plus 1pt minus 0.5pt
\parsep 2pt plus 1pt minus 0.5pt
\leftmargin 10pt \leftmargini 13pt \leftmarginii 10pt \leftmarginiii 5pt \leftmarginiv 5pt \leftmarginv 5pt \leftmarginvi 5pt
\labelwidth\leftmargini\advance\labelwidth-\labelsep \labelsep 5pt
\def\@listi{\leftmargin\leftmargini}
\def\@listii{\leftmargin\leftmarginii
\labelwidth\leftmarginii\advance\labelwidth-\labelsep
\topsep 2pt plus 1pt minus 0.5pt
\parsep 1pt plus 0.5pt minus 0.5pt
\itemsep \parsep}
\def\@listiii{\leftmargin\leftmarginiii
\labelwidth\leftmarginiii\advance\labelwidth-\labelsep
\topsep 1pt plus 0.5pt minus 0.5pt
\parsep \z@
\partopsep 0.5pt plus 0pt minus 0.5pt
\itemsep \topsep}
\def\@listiv{\leftmargin\leftmarginiv
\labelwidth\leftmarginiv\advance\labelwidth-\labelsep}
\def\@listv{\leftmargin\leftmarginv
\labelwidth\leftmarginv\advance\labelwidth-\labelsep}
\def\@listvi{\leftmargin\leftmarginvi
\labelwidth\leftmarginvi\advance\labelwidth-\labelsep}
\abovedisplayskip 7pt plus2pt minus5pt%
\belowdisplayskip \abovedisplayskip
\abovedisplayshortskip 0pt plus3pt%
\belowdisplayshortskip 4pt plus3pt minus3pt%


\bibliographystyle{abbrvnat}
\setcitestyle{authoryear,open={(},close={)}}

\newcommand{\R}{\ensuremath{\mathbb{R}}}
\newcommand{\Pc}{\ensuremath{\mathcal{P}}}
\newcommand{\slack}{\ensuremath{\xi}}
\newcommand{\slacks}{\ensuremath{\bm{\xi}}}
\newcommand{\weight}{\ensuremath{\beta}}
\newcommand{\weights}{\ensuremath{\bm{\beta}}}

\newtheorem{definition}{Definition}
\newtheorem{lemma}{Lemma}

\newcommand\numberthis{\addtocounter{equation}{1}\tag{\theequation}}

\title{Theory for Rank2Plan}

\begin{document}

\maketitle

\section{First Order Methods}

While the linear programs we seek to solve for ranking are similar to that
\citet{Dedieu2022}, there are a few important differences. This in particular
means we have to adapt the first order methods used by \citet{Dedieu2022} to
find an initial low-accuracy solution. The following discussion is based on
Section 4 of \citet{Dedieu2022}.

The ranking problem can be described as given a set of feature vectors $X \in
    \R^{n \times p}$, and a set of pairs $P = \{(i, j)\}$,
find a ranking function $r: \R^p \rightarrow \R$ such that $r(X_j) - r(X_i)
    \ge g_{i,j} \in \R_{\ge}$ with importance $s_{i,j} \in \R_{\ge}$.\footnote{Note that the order of $i$
    and $j$ means we prefer lower values} We restrict $r$ to be a linear
function with zero bias, in which case the ranking problem can be formulated
as the following linear program
\begin{align*}
    \Pc_\lambda := \min_{\slacks \in \R^m} & \sum_{(i,j) \in P} s_{i, j} \slack_{i,j} + \lambda \sum_{i = 1}^p \weight^+_i + \lambda \sum_{i=1}^p \weight^-_i                                 \\
    \text{s.t.}                            & \weights^+ X_j^T - \weights^- X_j^T - \weights^+ X_i^T + \weights^- X_i^T \ge g_{i,j} - \slack_{i,j} \ \ (i, j) \in P \numberthis \label{eqn:lp} \\
                                           & \slacks \ge 0, \weights^+ \ge 0, \weights^- \ge 0
\end{align*}
where $m = |P|$ and $r$ is given by $r(\bm{x}) = \weights \bm{x}^T$ where $\weights =
    \weights^+ - \weights^-$.

\subsection{Solving the Composite Form with Nesterov's Smoothing}

For scaler $u$, we have that $\max(0, u) = \frac12 (u + |u|) = \max_{|w|
        \le 1} \frac12 (u + wu)$ with the maximum achieved when $w =
    \text{sign}(u)$. Using this, the hinge loss in $\Pc_\lambda$ can thus be expressed
as
\begin{equation}
    \sum_{(i,j) \in P} (z_{i,j})_+ = \max_{\lVert \bm{w} \rVert_\infty \le 1} \sum_{(i,j) \in P} \frac12 [z_{i,j} + w_{i,j} z_{i,j}], \label{eqn:hinge-loss-expanded}
\end{equation}
where $z_{i,j} = s_{i,j}(g_{i,j} - (X_j^T \weights - X_i^T \weights))$. At this
point, it is tempting to just divide by $s_{i,j} g_{i,j}$ to retrieve the same
form as used in \citet{Dedieu2022}, which would make everything really easy!
Unfortunately, to preserve the original solution to the LP, we cannot divide
different terms by a different value and so we must do the work ourselves.

One can obtain a smooth approximate of \eqref{eqn:hinge-loss-expanded}
as\footnote{We abuse the notation slightly and treat $\bm{w}$ and $\bm{z}$ as
    vectors even though they are indexed by the pair $(i, j)$.}
\begin{equation}
    H^\tau(\bm{z}) := \max_{\lVert \bm{w} \rVert_\infty \le 1} \sum_{(i,j) \in P} \frac12 [z_{i,j} + w_{i,j} z_{i,j}] - \frac{\tau}{2} \lVert w \rVert_2^2,
\end{equation}
where $\tau$ controls the smoothness of $H^\tau$ and how well it approximates
the original hinge loss (where $\tau=0$).

The following lemma, which is lemma 7 from \citet{Dedieu2022} and originally from
\citet{Nesterov2005}, characterises $H^\tau$.
\begin{lemma}\label{lem:smooth-approximation}
    The function $\bm{z} \mapsto H^\tau(\bm{z})$ is an $O(\tau)$ approximation
    for the hinge loss $H^0(\bm{z})$, i.e. $H^0(\bm{z}) \in [H^\tau(\bm{z}),
            H^\tau(\bm{z}) + n\tau / 2]$ for all $\bm{z}$. Furthermore, $H^\tau(\bm{z})$
    has Lipschitz continuous gradient with parameter $1/(4\tau)$, i.e. $\lVert
        \nabla H^\tau(\bm{z}) - \nabla H^\tau(\bm{z'}) \rVert \le 1/(4\tau) \lVert \bm{z} - \bm{z'} \rVert_2$.
\end{lemma}

Let us define,
\begin{equation}
    F^\tau(\weights) = \max_{\lVert \bm{w} \rVert_\infty \le 1} \left\{ \sum_{(i, j) \in P} \frac12 [s_{i,j}(g_{i,j} - (X_j^T \weights - X_i^T \weights)) + w_{i,j}s_{i,j}(g_{i,j} - (X_j^T \weights - X_i^T \weights)) ] - \frac{\tau}{2} \lVert \bm{w} \rVert_2^2 \right\}. \label{eqn:f-tau}
\end{equation}
By Lemma \ref{lem:smooth-approximation}, $F^\tau$ is a uniform
$O(\tau)$-approximation of the hinge loss function. Its gradient is given by,
\begin{equation}
    \nabla F^\tau(\weights) = -\frac12 \sum_{(i, j) \in P} (1 + w_{i,j}^\tau) s_{i,j} (X_j - X_i) \in \R^p \label{eqn:f-tau-derivative}
\end{equation}
where $\bm{w}^\tau$ is the optimal solution to (\ref{eqn:f-tau}) at $\weights$.
Further, $\weights \mapsto \nabla F^\tau(\weights)$ is Lipschitz-continuous with
parameter $C^\tau = \sigma_{\text{max}}(\tilde{X}^T \tilde{X}) / (4 \tau)$ where
$\tilde{X}$ is the matrix with rows given by $s_{i, j} (X_{j} - X_{i})^T$ for $(i,
    j) \in P$ and $\sigma_{\text{max}}$ denotes the maximum eigenvalue as
\begin{align*}
    \lVert \bm{z} - \bm{z'} \rVert_2^2 & =\sum_{(i, j) \in P} [s_{i, j} (g_{i, j} - (X_j^T \weights - X_i^T \weights)) - (s_{i, j} (g_{i, j} - (X_j^T \weights' - X_i^T \weights')))]^2 \\
                                       & =\sum_{(i, j) \in P} [s_{i, j} (X_j - X_i)^T(\weights' - \weights)]^2                                                                          \\
                                       & =\sum_{(i, j) \in P} [\tilde{X}_{i,j}^T(\weights' - \weights)]^2                                                                               \\
                                       & =\lVert \tilde{X} (\weights - \weights') \rVert_2^2                                                                                            \\
                                       & \le \lVert \tilde{X} \rVert_2^2 \lVert \weights - \weights' \rVert_2^2                                                                         \\
                                       & = \sigma_{\text{max}} (\tilde{X}^T \tilde{X}) \lVert \weights - \weights' \rVert_2^2
\end{align*}

The rest then follows directly in the exact same logic as Section 4 of \citet{Dedieu2022}.

\bibliography{theory}

\end{document}